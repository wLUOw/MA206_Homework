\documentclass{article}

% Language setting
% Replace `english' with e.g. `spanish' to change the document language
\usepackage[UTF8]{ctex}

% Set page size and margins
% Replace `letterpaper' with `a4paper' for UK/EU standard size
\usepackage[letterpaper,top=2cm,bottom=2cm,left=3cm,right=3cm,marginparwidth=1.75cm]{geometry}

% Useful packages
\usepackage{amsmath}
\usepackage{graphicx}
\usepackage[colorlinks=true, allcolors=blue]{hyperref}

\title{MA206 Homework8}
\author{12110120 赵钊}
\date{}

\begin{document}
\maketitle


\section{第3题}
运用单纯形表法。

\subsection{a}
\begin{table}[!ht]
    \centering
    \begin{tabular}{|c|c|c|c|c|c|c|c|}
    \hline
         & x & y & a & b & c &  &  \\ \hline
        a & 8 & 6 & 1 & 0 & 0 & 48 & 48/6=8 \\ \hline
        b & 4 & 1 & 0 & 1 & 0 & 20 & 20/1=20 \\ \hline
        c & 0 & -1 & 0 & 0 & 1 & -5 & -5/-1=5 \\ \hline
         & 10 & 35 & 0 & 0 & 0 & z=0 &  \\ \hline
    \end{tabular}
\end{table}

\begin{table}[!ht]
    \centering
    \begin{tabular}{|c|c|c|c|c|c|c|c|}
    \hline
         & x & y & a & b & c &  &  \\ \hline
        a & $\frac{4}{3}$ & 0 & $\frac{1}{6}$ & 0 & 1 & 3 & 3/1=3 \\ \hline
        b & 4 & 0 & 0 & 1 & 1 & 15 & 15/1=15 \\ \hline
        y & 0 & 1 & 0 & 0 & -1 & -5 & -5/-1=5 \\ \hline
         & 10 & 0 & 0 & 0 & 35 & z=-175 &  \\ \hline
    \end{tabular}
\end{table}

\begin{table}[!ht]
    \centering
    \begin{tabular}{|c|c|c|c|c|c|c|c|}
    \hline
         & x & y & a & b & c &  &  \\ \hline
        c & $\frac{4}{3}$ & 0 & $\frac{1}{6}$ & 0 & 1 & 3 & / \\ \hline
        b & $\frac{8}{3}$ & 0 & $-\frac{1}{6}$ & 1 & 0 & 12 & / \\ \hline
        y & $\frac{4}{3}$ & 1 & $\frac{1}{6}$ & 0 & 0 & 8 & / \\ \hline
         & $-\frac{110}{3}$ & 0 & $-\frac{35}{6}$ & 0 & 0 & z=280 &  \\ \hline
    \end{tabular}
\end{table}

z的最大值为280,此时x=0,y=8


\subsection{b}
同(a)的方法,z最小值为14,此时x=0,y=2

\newpage

\section{第4题}

\subsection{a}
运用单纯形表法。2x+3y的最小值为6,此时x=0,y=2.最大值为21,此时x=6,y=3.

\subsection{b}
运用单纯形表法。6x+4y的最小值为0,此时x=0,y=0.最大值为144,此时x=24,y=0.


\end{document}
