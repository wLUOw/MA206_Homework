\documentclass{article}

% Language setting
% Replace `english' with e.g. `spanish' to change the document language
\usepackage[UTF8]{ctex}

% Set page size and margins
% Replace `letterpaper' with `a4paper' for UK/EU standard size
\usepackage[letterpaper,top=2cm,bottom=2cm,left=3cm,right=3cm,marginparwidth=1.75cm]{geometry}

% Useful packages
\usepackage{amsmath}
\usepackage{graphicx}
\usepackage[colorlinks=true, allcolors=blue]{hyperref}

\title{MA206 Homework3}
\author{12110120 赵钊}

\begin{document}
\maketitle


\section{概述}

这是MA206 2023 Spring的第3次作业,共2题。

第1题为利用动力系统计算总经济能否达到目标及达到目标的时间,解题步骤为列出递推式,通过递推式求解出通式,最后进行判断。

第2题与第1题相似,列出递推式并进行求解。第2问通过计算总还款金额来判断重新融资是否合理。

\section{第1题}

\subsection{符号说明}

\begin{table}[!h]
\begin{center}
\begin{tabular}{|c|c|c|}
    \hline
    序号 & 符号 & 符号说明\\
    \hline
    1 & $a_n$ & 第$n$个月后需要还款的金额\\
    \hline
    2 & $r_m$ & 第$m$年增加的利息\\ 
    \hline
    3 & $l$ & 最优惠成交额\\ 
    \hline
    4 & $l_0$ & 定金\\ 
    \hline
\end{tabular}
\caption{\label{demo-table}第1题符号说明}
\end{center}
\end{table}

\subsection{递推式分析}
根据题意,容易得知,在支付完定金之后,有关系式:
\[a_n = a_0 - 500 + \frac{1}{12}r_m\]

其中
\[a_0 = l - l_0\]

\subsection{问题求解}
以列表中的第1个Ford Fiesta为例,在第1个月支付完定金500\$后,
\[a_0 = 14200 - 500 = 13700\]
\[r_1 = a_0 \times 4.5\% = 616.5\]

因此第1年过后,
\[a_12 = a_0 - 6000 + r_1 = 8316.5\]
\[r_2 = a_12 \times 4.5\% = 374.2425\]

第2年过后,
\[a_24 = a_12 - 6000 + r_2 = 2690.7425\]
\[r_2 = a_24 \times 4.5\% = 121.01\]

发现
\[a_36 = a_24 - 6000 + r_3 < 0\]

得出结论在付清定金的第3年内,可以买到Ford Fiesta

同样的方式对其他7个公司提供的方案进行求解,得到结果如下
\begin{table}[!h]
\begin{center}
\begin{tabular}{|c|c|}
    \hline
    名称 & 能否购买\\
    \hline
    Ford Focus & 能 \\
    \hline
    Chevy Volt & 否 \\ 
    \hline
    Chevy Cruz & 能 \\ 
    \hline
    Toyota Camry & 能 \\ 
    \hline
    Toyota Camry Hybrid & 能\\ 
    \hline
    Toyota Corolla & 能 \\ 
    \hline
    Toyota Prius & 能 \\ 
    \hline
\end{tabular}
\caption{\label{demo-table}第1题结论}
\end{center}
\end{table}

\subsection{结论}
按照题目条件给定的策略进行购买,可以买到Ford Fiesta、Ford Focus、Chevy Cruz、Toyota Camry、Toyota Camry Hybrid、Toyota Corolla和Toyota Prius,即除去Chevy Volt均可以购买。


\section{第2题}

\subsection{符号说明}

\begin{table}[!h]
\begin{center}
\begin{tabular}{|c|c|c|}
    \hline
    序号 & 符号 & 符号说明\\
    \hline
    1 & $a_n$ & 第$n$个月后需要还款的金额\\
    \hline
    2 & $r$ & 月利率\\ 
    \hline
    3 & $p$ & 月还款额\\  
    \hline
\end{tabular}
\caption{\label{demo-table}第2题符号说明}
\end{center}
\end{table}

\subsection{第1问}
由题目可知:
\[a_0 = 250000, r = 0.4\%\]

根据等量关系列出递推式:
\[a_n = (1+r)a_{n-1} - p\]

进行恒等变换有:
\[a_n - \frac{p}{r} = (1+r)a_{n-1} - p - \frac{p}{r}\]
\[a_n - \frac{p}{r} = (1+r)(a_{n-1} - \frac{p}{r})\]

由递推式可以得到:
\[a_n = (1+r)^n(a_0 - \frac{p}{r}) + \frac{p}{r}\]

带入$a_{360} = 0$、$a_0 = 250000$和$r=0.4\%$解得:
\[p=1311.66\]

结论:每月还款数额$p=1311.66\$$


\subsection{第2问}
\[a_{96} = (1+r)^{96}(a_0-\frac{p_0}{r}) + \frac{p_0}{r}\]

其中
\[a_0 = 250000, r = 0.4\%, p_0 = 1311.66\]

得到
\[a_{96} = 213614\]

同样地,将$b_0=213614$、$b_{20}=0$、$r_1=4\%$代入如下表达式
\[b_n=(1+r_1)^n(b_0-\frac{p_1}{r_1})+\frac{p_1}{r_1}\]

解得
\[p_1=15718.2\]

因此改为20年期贷款的月付额为$\frac{1}{12}p_1=1309.85\$$

同法可计算出,15年期贷款的月付额为1578.37\$

原总支付额为$s_0 = 1311.66 \times 360 = 472198\$$

20年期贷款总支付额为$s_1 = 1311.66 \times 96 + 1309.85 \times 240 + 2500 = 442783\$$

15年期贷款总支付额为$s_2 = 1311.66 \times 96 + 1578.37 \times 180 + 2500 = 412526\$$

发现$s_2 < s_1 < s_0$

因此应该选择进行融资,如果经济可以承担15年期贷款的月付额,应优先选择15年期贷款.


\end{document}