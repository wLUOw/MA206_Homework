\documentclass{article}

% Language setting
% Replace `english' with e.g. `spanish' to change the document language
\usepackage[UTF8]{ctex}

% Set page size and margins
% Replace `letterpaper' with `a4paper' for UK/EU standard size
\usepackage[letterpaper,top=2cm,bottom=2cm,left=3cm,right=3cm,marginparwidth=1.75cm]{geometry}

% Useful packages
\usepackage{amsmath}
\usepackage{graphicx}
\usepackage[colorlinks=true, allcolors=blue]{hyperref}

\title{MA206 Homework6}
\author{12110120 赵钊}
\date{}

\begin{document}
\maketitle


\section{第1题}
假设$P$、$Q$是两个马尔可夫矩阵,$R=PQ$,现在证明$R$也是马尔可夫矩阵。
设$(\star)_{ij}$表示矩阵$(\star)$第$i$行第$j$列的元素。\\
由$P$、$Q$是马尔可夫矩阵,有
\[p_{ij} \geq 0 \quad q_{ij} \geq 0\]
\[\sum\limits_j p_{ij} =1 \quad \sum\limits_j q_{ij} =1\]
那么,
\[r_{ij} = \sum\limits_{k=1}^n p_{ik}q_{kj}  \geq0\]
\begin{equation*}
\begin{split}
\sum\limits_j r_{ij}
 &= \sum\limits_{k=1}^n p_{ik}q_{k1} + \sum\limits_{k=1}^n p_{ik}q_{k2} + \dots + \sum\limits_{k=1}^n p_{ik}q_{kn} \\
 &= p_{i1}\sum\limits_{k=1}^n q_{1k} + p_{i2}\sum\limits_{k=1}^n q_{2k} + \dots + p_{in}\sum\limits_{k=1}^n q_{nk} \\
 &= p_{i1} + p_{i2} + \dots + p_{in} \\
 &= 1
\end{split}
\end{equation*}
根据定义,$R$也是马尔可夫矩阵 \\
因此由数学归纳法,$P$是马尔可夫矩阵,$P^2=P\cdot P$为马尔可夫矩阵,$P^3=P^2\cdot P$为马尔可夫矩阵,......
也就是,若$P$是马尔可夫矩阵,则$\forall n$,$P^n$是马尔可夫矩阵


\section{第2题}
只需要等价的证明从$i$到$j$的$m$步转移概率为$P_{ij}^m$,也就是
\[P(X_{n+m}=j|X_n=i)=p^m(i,j)=P_{ij}^m\]

首先根据条件概率公式,有
\[P(X_{n+m}=j|X_n=i) = \frac{P(X_{n+m}=j,X_n=i)}{P(X_n=i)}\]
对于分子$P(X_{n+m}=j,X_n=i)$,中间经过了$m$步,因此一个简单的想法就是把这$m$步的所有的可能的情况都列出来,也就是
\[P(X_{n+m}=j,X_n=i) = \sum\limits_{i_1,\dots,i_{m-1}\in S}P(X_{n+m}=j,X_{n+m-1}=i_{m-1},\dots,X_n=i)\]
这里的$S$表示的是所有的状态的集合.\\
考虑全概率公式,即
\[\]
\begin{equation*}
\begin{split}
LHS
&= P(X_{n+m}=j,X_{n+m-1}=i_{m-1},\dots,X_n=i) \\
 &= P(X_n=i) P(X_{n+1}=i_1|X_n=i) \dots P(X_{n+m}=j|X_{n+m-1}=i_{m-1},X_{n+m-2}=i_{m-2},\dots) \\
\end{split}
\end{equation*}
又由于每个$X_i$都有马尔科夫性,得到
\[P(X_{n+m}=j|X_{n+m-1}=i_{m-1},X_{n+m-2}=i_{m-2},\dots,X_n = i)=P(X_n=i)p(i,i_1)p(i_1,i_2)\dots p(i_{m-1},j) \]
代入求和得
\begin{equation*}
\begin{split}
P(X_{n+m}=j,X_n=i)
&= P(X_n=i)\sum\limits_{i_1,\dots,i_{m-1}\in S}p(i,i_1)p(i_1,i_2)\dots p(i_{m-1},j) \\
 &= P(X_n=i)p^m(i,j) \\
\end{split}
\end{equation*}
因此结论得证

\section{文献报告}

\subsection{Pagerank概述}

Pagerank,即网页排名,是Google创始人拉里·佩奇和谢尔盖·布林于1997年构建早期的搜索系统原型时提出的链接分析算法,自从Google在商业上获得空前的成功后,该算法也成为其他搜索引擎和学术界十分关注的计算模型。目前很多重要的链接分析算法都是在PageRank算法基础上衍生出来的。PageRank是Google用于用来标识网页的等级/重要性的一种方法,是Google用来衡量一个网站的好坏的唯一标准。在揉合了诸如Title标识和Keywords标识等所有其它因素之后,Google通过PageRank来调整结果,使那些更具“等级/重要性”的网页在搜索结果中另网站排名获得提升,从而提高搜索结果的相关性和质量。

\subsection{基本假设}

PageRank的计算基于以下两个基本假设:
\begin{enumerate}
    \item 数量假设:在Web图模型中,如果一个页面节点接收到的其他网页指向的入链数量越多,那么这个页面越重要。
    \item 质量假设:指向页面A的入链质量不同,质量高的页面会通过链接向其他页面传递更多的权重。所以越是质量高的页面指向页面A,则页面A越重要。
    
\end{enumerate}

利用以上两个假设,PageRank算法刚开始赋予每个网页相同的重要性得分,通过迭代递归计算来更新每个页面节点的PageRank得分,直到得分稳定为止。 PageRank计算得出的结果是网页的重要性评价,这和用户输入的查询是没有任何关系的,即算法是主题无关的。假设有一个搜索引擎,其相似度计算函数不考虑内容相似因素,完全采用PageRank来进行排序,这个搜索引擎对于任意不同的查询请求,返回的结果都是相同的,即返回PageRank值最高的页面。

\subsection{基本思想}
如果网页T存在一个指向网页A的连接,则表明T的所有者认为A比较重要,从而把T的一部分重要性得分赋予A。这个重要性得分值为:$\frac{PR(T)}{L(T)}$.其中PR(T)为T的PageRank值,L(T)为T的出链数,则A的PageRank值为一系列类似于T的页面重要性得分值的累加。

即一个页面的得票数由所有链向它的页面的重要性来决定,到一个页面的超链接相当于对该页投一票。一个页面的PageRank是由所有链向它的页面(链入页面)的重要性经过递归算法得到的。一个有较多链入的页面会有较高的等级,相反如果一个页面没有任何链入页面,那么它没有等级。


\subsection{优缺点分析}
\subsubsection{优点}
\begin{enumerate}
    \item Pagerank是一个与查询无关的静态算法,所有网页的PageRank值通过离线计算获得;有效减少在线查询时的计算量,极大降低了查询响应时间。
\end{enumerate}

\subsubsection{缺点}
\begin{enumerate}
    \item 查询具有主题特征,PageRank忽略了主题相关性,导致结果的相关性和主题性降低。
    \item 旧的页面等级会比新页面高。因为即使是非常好的新页面也不会有很多上游链接,除非它是某个站点的子站点。
\end{enumerate}







\end{document}
