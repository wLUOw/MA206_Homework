\documentclass{article}

% Language setting
% Replace `english' with e.g. `spanish' to change the document language
\usepackage[UTF8]{ctex}

% Set page size and margins
% Replace `letterpaper' with `a4paper' for UK/EU standard size
\usepackage[letterpaper,top=2cm,bottom=2cm,left=3cm,right=3cm,marginparwidth=1.75cm]{geometry}

% Useful packages
\usepackage{amsmath}
\usepackage{graphicx}
\usepackage[colorlinks=true, allcolors=blue]{hyperref}

\title{MA206 Homework9}
\author{12110120 赵钊}
\date{}

\begin{document}
\maketitle


\section{第1题}
根据万有引力公式
\[\frac{Gm}{r^2} = \omega^2 r\]
以及
\[\omega = \frac{2\pi}{t}\]
可以解出
\[t = \frac{2\pi}{\sqrt{G}} m^{-\frac{1}{2}} r^\frac{3}{2}\]
因此可以得到$a = -\frac{1}{2}$、$b = \frac{3}{2}$和$c = -\frac{1}{2}$

\section{第2题}
$\theta$和$sin\theta$的量纲为1,$g$是加速度的量纲,为$LT^{-2}$ \\
因此,等式右边的量纲为
\[\left[ ML^2\left(T^{-2}\right)MLT^{-2}L \right] T^{-1} = M^2L^4T^{-5}\]
而能量$E$的量纲为$ML^2T^{-2}$,因此量纲不匹配,该公式不能成立。

\section{第3题}
计算等式右端用到的量纲如下表:
\begin{table}[!ht]
    \centering
    \begin{tabular}{|l|l|}
    \hline
        物理量 & 量纲 \\ \hline
        $\rho$ & $ML^{-3}$ \\ \hline
        $v$ & $LT^{-1}$ \\ \hline
        $r$ & $L$ \\ \hline
        $\delta$ & $L$ \\ \hline
        $\theta$ & 1 \\ \hline
        $\mu$ & $ML^{-1}T^{-1}$ \\ \hline
        $s$ & $LT^{-1}$ \\ \hline
        $g$ & $LT^{-2}$ \\ \hline
    \end{tabular}
\end{table}

\newpage

等式右边的量纲为
\[ML^{-3}(LT^{-1})^2L^2h(1,1,1,1,1) = MLT^{-2}\]

而$F$的量纲也为$MLT^{-2}$,因此要证等式在量纲上成立。

\end{document}
